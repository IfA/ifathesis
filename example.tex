\documentclass[
%  print,         %print optimized version of the thesis, standard option
                 % alternative:
  screen,        % makes the thesis better readable on screens (onesided, coloured links)
                 % use only 'print' OR 'screen'
                 % ATTENTION: 'screen' hanges size of text area slightly. Always optimize document WITH
                 % 'print' option first for printing (size of graphics etc.) and convert afterwards
                 % in screen optimized version!
  listoffigures, % includes list of figures
  listoftables,  % includes list of tables
  abbrevations   % includes index of symbols and abbrevations
]{ifathesis}

\ifaThesis{Diplomarbeit}
\ifaAuthor{Heinrich Grabow}
\ifaAuthorBirthday{20.02.1980}
\ifaAuthorBirthplace{Lutherstadt Eisleben}
\ifaAuthorCourse{Mechatronik}
\ifaAuthorYearOfMatriculation{2002}
\ifaKeywords{Diplomarbeit, Web, SPS, Security, XML} % Keywords included in pdf-file. Could be found e.g. by Windows file search.
\ifaTitleDE{Teleautomation, Chancen und Risiken}
\ifaTitleEN{Teleautomation, Chances and Risks}
\ifaSupervisorA{Dipl.-Ing. Stefan Hennig}
\ifaSupervisorB{Dipl.-Ing. Arne Sonnenburg}
\ifaSupervisorC{PD Dr.-Ing. Annerose Braune}
%\ifaSupervisorD{}
%\ifaSupervisorE{}
\ifaProfessor{Prof. Dr. techn. Klaus Janschek}
\ifaDayOfSubmission{21.03.2009}
\ifaTopicDescriptionPDF{example_files/00_Aufgabenstellung.pdf}
\ifaAppendix{example_files/appendix}
\ifaAbstractDE{example_files/00_abstract_de}
\ifaAbstractEN{example_files/00_abstract_en__invalid}  % HINWEIS: Der Dateiname der englischen Kurzfassung ist absichtlich
                                                       % ung�ltig gemacht worden, um zu demonstrieren, dass die Vorlage
                                                       % auch ohne Kurzfassungen genutzt werden kann. Dazu einfach einen
                                                       % ung�ltigen Dateinamen angeben oder den Befehl ganz weglassen.
                                                       % 
                                                       % ACHTUNG: Im Institut f�r Automatisierungstechnik sind die 
                                                       % Kurzfassungen unbedingt erforderlich. Eine Arbeit ohne 
                                                       % Kurzfassung wird nicht angenommen, da sie nicht den Richtlinien
                                                       % des Instituts entspricht.
\ifaAbbrev{example_files/00_Abbrev}
\ifaReferences{bibliography}


\begin{document}

%!TEX root = ../example.tex
%*******************************************************************************
% * Copyright (c) 2006-2013 
% * Institute of Automation, Dresden University of Technology
% * 
% * All rights reserved. This program and the accompanying materials
% * are made available under the terms of the Eclipse Public License v1.0 
% * which accompanies this distribution, and is available at
% * http://www.eclipse.org/legal/epl-v10.html
% * 
% * Contributors:
% *   Institute of Automation - TU Dresden, Germany 
% *      - initial API and implementation
% ******************************************************************************/

%%%%%%%%%%%%%%%%%%%%%%%%%%%%%%%%%%%%%%%%%%%%%%%%%%%%%%%%%%%%%%%%%%%%%%
%%%%%%%%%%%%%%%%%%%%%%%%%%%%%%%%%%%%%%%%%%%%%%%%%%%%%%%%%%%%%%%%%%%%%%
\chapter{Verbindlichkeiten vorab}
\label{sec:VerbindlichkeitenVorab}

\begin{compactitem}
  \item Verbindliche Grundlage ist die zu Beginn der Diplom-/Studienarbeit gültige \href{http://www.et.tu-dresden.de/ifa/index.php?id=330}{IfA-Richtlinie}.
  \item Die \href{http://www.et.tu-dresden.de/etit/uploads/media/EmpfehlungWissenschArbeiten2013_05.pdf}{Empfehlung der Fakultät Elektrotechnik und Informationstechnik für die Ausarbeitung wissenschaftlicher Arbeiten} (Studienarbeiten oder Diplomarbeiten) ist als Ergänzung gedacht, im Konfliktfall ist die IfA-Richtlinie anzuwenden.
  \item Diplom-/Studienarbeit sind in Absprache mit dem Betreuer gemäß dem \href{http://www.et.tu-dresden.de/ifa/index.php?id=331&L=1?keepThis=true}{IfA-Vorgehensmodell} abzuwickeln.
  \item Für die Inhalte einer Studien-/Diplomarbeit gilt die \href{http://www.et.tu-dresden.de/ifa/index.php?id=336&L=1?keepThis=true}{Richtlinie} des Instituts für Automatisierungstechnik, die als Beispieltext auch in \autoref{sec:IfARichtlinieFuerWissenschaftlicheUndStudentischeArbeiten} dieses Dokuments zu finden ist
\end{compactitem}
%!TEX root = ../example.tex
%*******************************************************************************
% * Copyright (c) 2006-2013 
% * Institute of Automation, Dresden University of Technology
% * 
% * All rights reserved. This program and the accompanying materials
% * are made available under the terms of the Eclipse Public License v1.0 
% * which accompanies this distribution, and is available at
% * http://www.eclipse.org/legal/epl-v10.html
% * 
% * Contributors:
% *   Institute of Automation - TU Dresden, Germany 
% *      - initial API and implementation
% ******************************************************************************/

%%%%%%%%%%%%%%%%%%%%%%%%%%%%%%%%%%%%%%%%%%%%%%%%%%%%%%%%%%%%%%%%%%%%%%
%%%%%%%%%%%%%%%%%%%%%%%%%%%%%%%%%%%%%%%%%%%%%%%%%%%%%%%%%%%%%%%%%%%%%%
\chapter{Allgemeine Hinweise zur Vorlage}
\label{sec:AllgemeineHinweiseZurVorlage}

%%%%%%%%%%%%%%%%%%%%%%%%%%%%%%%%%%%%%%%%%%%%%%%%%%%%%%%%%%%%%%%%%%%%%%
\section{Installation}
\label{sec:install}
%%%%%%%%%%%%%%%%%%%%%%%%%%%%%%%%%%%%%%%%%%%%%%%%%%%%%%%%%%%%%%%%%%%%%%

Die Vorlage \verb-ifathesis- ist eine fertige \LaTeX-Klasse, die die Richtlinie für wissenschaftliche Arbeiten am \emph{Institut für Automatisierungstechnik, Technische Universität Dresden} (IfA) realisiert. Um diese Vorlage zu verwenden, müssen die folgenden Dateien lediglich im Wurzelverzeichnis des zu verfassenden \LaTeX-Dokuments liegen:
\nomenclature[ba ]{IfA}{Institut für Automatisierungstechnik}
\begin{compactitem}
  \item \verb-ifathesis.cls-
  \item \verb-packages.tex-
  \item \verb-TU_Logo_SW.pdf-
  \item \verb-IfA_Logo_SW.pdf-
  \item \verb-IfA_Header.pdf-
  \item \verb-nomencl.cfg-
  \item \verb-nomencl.ist-
\end{compactitem}
Alternativ können diese Dateien dem \LaTeX-Paketbaum hinzugefügt werden. Dann steht die Vorlage stets zur Verfügung. Die dafür notwendigen Schritte sind allerdings von der verwendeten \LaTeX-Distribution sowie vom Betriebssystem abhängig, daher wird an dieser Stelle auf eine Internetrecherche verwiesen: \href{http://www.google.de/search?sourceid=chrome&ie=UTF-8&q=latex+install+packages}{z.B. über Google.}

Zu beachten ist\marginpar{biblatex}, dass unter Windows mindestens die MikTex-Version 2.9 benötigt wird. Unter Linux oder Mac OS X (texlive-Distributionen) muss möglicherweise das Paket \href{http://ctan.org/tex-archive/macros/latex/contrib/biblatex}{biblatex} manuell aktualisiert werden. Zu empfehlen ist dann, auch die Pakete \href{http://ctan.org/tex-archive/macros/latex/contrib/logreq}{logreq} sowie \href{http://ctan.org/tex-archive/macros/latex/contrib/etoolbox}{etoolbox} in einer aktuellen Version zu installieren.

Ein weiterer wichtiger Punkt ist die Dateikodierung\marginpar{encoding}. Im Test der Vorlage konnten oft Diskrepanzen bei der Darstellung der Sonderzeichen festgestellt werden. Das lag an verschiedenen Zeichenkodierungen, mit der die Dateien abgespeichert werden. Daher haben wir uns für die Vorlage auf die Kodierung \emph{UTF-8} geeinigt.

Um die automatische Generierung eines Abkürzungs- und Symbolverzeichnises zu benutzen, müssen Sie zusätzlich zum pdftex-Kompiler den makeindex-Kompiler benutzen. Dieser ist entweder manuell oder automatisch (eingebunden im verwendeten Entwicklungswerkzeug) aufzurufen. Wie der makeindex-Kompiler eingebunden wird, ist jeweils abhängig vom genutzten Werkzeug. Für TeXnicCenter funktioniert dies wie folgt:

\noindent \begin{center} \verb|makeindex.exe "%tm.nlo" -s nomencl.ist -o "%tm.nls"|\footnote{Das ist ein Beispiel für das Betriebssystem Windows bei Verwendung von Miktex 2.9. Wie der Befehl auf anderen Betriebssystemen und \LaTeX-Distributionen genau lautet, entnehmt bitte einer Internetrecherche.}\end{center}

\noindent Dabei steht \verb|%tm| als Platzhalter für den Dateinamen der kompilierten .tex-Datei. Im TexnicCenter wird dieser Platzhalter automatisch durch den richtigen Dateinamen erstetzt.


%%%%%%%%%%%%%%%%%%%%%%%%%%%%%%%%%%%%%%%%%%%%%%%%%%%%%%%%%%%%%%%%%%%%%%
\section{Konfiguration}
%%%%%%%%%%%%%%%%%%%%%%%%%%%%%%%%%%%%%%%%%%%%%%%%%%%%%%%%%%%%%%%%%%%%%%

Dieser Vorlage liegt ein umfassendes Beispiel bei, welches mit der Vorlage erstellt wurde (siehe \verb-example.tex-) und
dieses PDF erzeugt. Bevor Sie eigenen Inhalt einfügen oder Anpassungen vornehmen, sollten Sie die Korrektheit Ihres
\LaTeX"=Setups testen: Stellen Sie dazu sicher, dass sich die Vorlage fehlerfrei übersetzen/bauen lässt und dass das
resultierende PDF äquivalent zu dieser Datei ist.


%%%%%%%%%%%%%%%%%%%%%%%%%%%%%%%%%%%%%%%%%%%%%%%%%
\subsection{Dokumentenklasse, Dokumentenoptionen}

Zur Verfügung steht die Dokumentenklasse \verb-ifathesis- mit folgenden Optionen:

\begin{description}
  \item[print | screen] Übersetzt das Dokument optimiert für den Druck bzw. für das Lesen am Monitor (einseitig, farbig hervorgehobene Links). \textbf{Achtung: Die Größe des Textkörpers unterscheidet sich geringfügig. Daher bitte immer zuerst das Dokument für den Druck optimert erstellen, wodurch bspw. die Größe der Grafiken richtig bestimmt werden kann. Anschließend kann in die für das Monitorlesen optimierte Version übersetzt werden}\footnote{Zur Abgabe der Arbeit auf einem Datenträger erforderlich}. Voreinstellung: \verb-print-.
  \item[listoffigures] Integriert ein Abbildungsverzeichnis an der richtigen Stelle, gemäß der \emph{IfA}-Richtlinie
  \item[listoftables] Integriert ein Tabellenverzeichnis an der richtigen Stelle, gemäß der \emph{IfA}-Richtlinie
  \item[listoftlistings] Integriert ein Quellcodeverzeichnis an der richtigen Stelle, gemäß der \emph{IfA}-Richtlinie
  \item[abbrevations] Integriert ein Symbol- und Abkürzungsverzeichnis an der richtigen Stelle, gemäß der \emph{IfA}-Richtlinie. Weitere Informationen finden Sie in Abschnitt \ref{sec:install} und in der Datei \verb-exmaple_files\00_abbrev.tex-.
  \item[bibIfa | bibNumeric | bibHarvard] Verwendung unterschiedlicher Stile für
  Zitate bzw. für das Literaturverzeichnis
  \item[bibtex | bibtex8 | biber] Festlegung des Bibliography"=Backends für das Paket \emph{biblatex} (relevant für das Format der verwendeten \texttt{*.bib}-Datei). Für Eigenheiten sei an dieser Stelle auf die \href{https://www.ctan.org/pkg/biblatex?lang=de}{Dokumentation} des \emph{biblatex}-Pakets verwiesen.
  \item[langDE | langEN] Festlegung der Sprache (wichtig für Beschriftungen, Silbentrennung etc.)
  \item[noIfaLogo] Verhindert die Einbindung des IfA"=Logos in die Kopfzeile von Titelseite und Abstract
  \item[a5paper] Schaltet auf A5-Format um. Darf nur bei Dissertationen verwendet werden\,--\,Diplomarbeiten, Studienarbeiten, etc. sind grundsätzlich im A4-Format zu drucken!!!
\end{description}


%%%%%%%%%%%%%%%%%%%%%%%%%%%%%%
\subsection{Weitere Parameter}

\begin{description}
  \item[ifaThesis] Art der Arbeit (gültige Werte \emph{Dissertation}, \emph{Diplomarbeit}, \emph{Masterarbeit}, \emph{Studienarbeit}, \emph{Final Project}, \emph{Bachelorarbeit}, \emph{Forschungspraktikum})
  \item[ifaAuthor] Name des Autors
  \item[ifaAuthorBirthday] Geburtsdatum des Autors (Format: \verb-TT.MM.JJJJ-)
  \item[ifaAuthorBirthplace] Geburtsort des Autors
  \item[ifaAuthorCourse] Studiengang des Autors
  \item[ifaAuthorYearOfMatriculation] Immatrikulationsjahr des Autors
  \item[ifaKeywords] Schlüsselwörter, die dem Thema zugeordnet werden können (als kommaseparierte Liste)
  \item[ifaTitleDE] Deutscher Titel der Arbeit
  \item[ifaTitleEN] Englische Übersetzung des Titels der Arbeit
  \item[ifaAbstractDE] Dateiname der deutschsprachigen Kurzfassung
  \item[ifaAbstractEN] Dateiname der englischsprachigen Kurzfassung
  \item[ifaAbbrev] Dateiname des Abkürzungsverzeichnisses
  \item[ifaUserListings] Dateiname einer Datei, die zusätzliche, selbst definierte Verzeichnisse druckt (z.\,B. ein
  Definitions"=Verzeichnis). Diese werden dann an der richtigen Stelle, gemäß der \emph{IfA}-Richtlinie, nach dem
  Abbildungs-, Tabellen- und Quellcodeverzeichnis ausgegeben.
  \item[ifaAcknowledgements] Dateiname der Danksagung
  \item[ifaSupervisor{\emph X}] mit $X=A\dots E$, Angabe von bis zu fünf Betreuern (A\,--\,E).
  \item[ifaProfessor] Betreuender Hochschullehrer
  \item[ifaDayOfSubmission] Tag der Einreichung
  \item[ifaTopicDescriptionPDF] Dateiname der Aufgabenstellung (als PDF). Es besteht damit die Möglichkeit, die Aufgabenstellung einzucannen und in die PDF-Version der Arbeit zu integrieren. Wird der Befehl weggelassen oder es ist ein ungültiger Dateiname angegeben, dann wird die Aufgabenstellung einfach ignoriert. \textcolor{red}{ACHTUNG: Am Institut für Automatisierungstechnik muss ein einzureichendes Exemplar der Arbeit die Aufgabenstellung im Original enthalten. Ansonsten führt dies zur Nichtannahme der Arbeit.}
  \item[ifaAppendix] Dateiname der Hauptdatei der Anhänge (vollständige Pfadangabe ausgehend von der Dokumentenwurzel erforderlich)
  \item[bibliography] Name der Bibliographiedatei (vollständige Pfadangabe
  ausgehend von der Dokumentenwurzel ohne Dateiendung erforderlich)
  \item[ifaBibliographyBeforeAppendix] Soll das Literaturverzeichnis vor oder nach dem Anhang aufgeführt werden? (mögliche Werte: true, false)
  \item[ifaAdditionalContributors] Standardmäßig werden bei der
  Selbstständigkeitserklärung die Betreuer als beitragende Personen angegeben.
  Sollen zusätzliche Personen aufgeführt werden, können diese hier definiert werden.
\end{description}


%%%%%%%%%%%%%%%%%%%%%%%%%%%%%%
\subsection{Weitere Parameter ausschließlich für Dissertationen}

\begin{description}
  \item[ifaDissertationStage] Nur bei Dissertationen zur Angabe der Phase der Arbeit (gültige Werte sind \emph{Gutachten} $\rightarrow$ Version für die Gutachter (vor der Verteidigung) und \emph{Pflichtexemplare} $\rightarrow$ Version für die Pflichtexemplare (nach der Verteidigung))
  \item[ifaChair] Vorsitzender der Prüfungskommission (nur für Dissertationen)
  \item[ifaDayOfDefense] Tag der Verteidigung (nur für Dissertationen)
  \item[ifaIncludeBeforeTitlePage] Dateiname für zusätzlichen Inhalt, der vor der Titelseite ausgegeben werden soll (nur bei Dissertationen; z.\,B. für Schmutztitel, Impressum, etc.)
  \item[ifaIncludeAfterTitlePage] Dateiname für zusätzlichen Inhalt, der zwischen Titelseite und Abstract ausgegeben werden soll (nur bei Dissertationen; z.\,B. für Impressum, Widmung, etc.)
  \item[ifaCV] Dateiname eines Lebenslaufs (als PDF), der am Ende der Arbeit eingebunden werden soll (nur bei Dissertationen)
  \item[ifaIncludeAtEndOfDocument] Dateiname für zusätzlichen Inhalt, der am Ende des Dokuments ausgegeben werden soll (nur bei Dissertationen; z.\,B. zum Erreichen einer geraden Seitenzahl)
\end{description}

%%%%%%%%%%%%%%%%%%%%%%%%%%%%%
\section{Neue Befehle und Umgebungen}
\label{sec:BefehleUndUmgebungen}

In der Vorlage wurden neue Befehle eingeführt, die meist als Wrapper für existierende Befehle gelten und diese so entsprechend der IfA-Richtlinie konfigurieren. Nachfolgend werden diese Befehle bzw. Umgebungen detailliert erläutert.

\subsection{ifalisting}
Hierbei handelt es sich um einen Befehl zur Erzeugung eines Quellcode-Listings, welcher wie folgt verwendet wird:

\begin{verbatim}
\ifalisting{<CAPTION>}
           {<LABEL>}
           {<LANGUAGE>}
           {<LINE_NUMBERS>}
           {<FILE>}
           {<FLOAT?>}
\end{verbatim}

\begin{description}
	\item[<CAPTION>] Die Überschrift des Listings.
	\item[<LABEL>] Erzeugt eine Marke, über die das Listing mit Hilfe des
	Befehls \verb|\ref{<LABEL>}| im Text referenziert werden kann.
	\item[<LANGUAGE>] Diese Angabe ist wichtig für die Syntaxhervorhebung. Latex kennt bspw. die Schlüsselwörter und ähnliches vieler Programmiersprachen, z.B.:
		\begin{compactitem}
			\item Assembler
			\item Java, C/C++
			\item Matlab
			\item OCL
			\item Phython, Perl, PHP, Ruby
			\item HTML, XML, XSLT
		\end{compactitem}
		Eine vollständige Liste ist unter anderem \href{http://en.wikibooks.org/wiki/LaTeX/Packages/Listings}{HIER} zu finden.
	\item[<LINE\_NUMBERS>] Anzeige einer Zeilennummerierung, gültige Werte: {\bfseries none|left|right}
	\item[<FILE>] Dateiname, in der der Quellcode bzw. das Quellcode-Fragment gespeichert ist. Listings sollten stets aus einer separaten Datei geladen werden. Bei mehreren Listings bietet es sich daher an, ein separates Verzeichnis dafür anzulegen.
	\item[<FLOAT?>] Mögliche Werte: \textbf{true}, \textbf{false}. Wenn \textbf{true}, dann wird das Listing als Gleitobjekt behandelt, es "`gleitet"' dann jeweils an die nächste freie Stelle. Gleitobjekte können allerdings nicht unterbrochen werden, das heißt, dass kein Seitenumbruch bei längeren Quelltexten möglich ist. In einem solchen Fall sollte auf \textbf{false} zurückgegriffen werden. Das Listing erscheint dann genau an der definierten Stelle, wird aber bei einem Seitenumbruch auf der nächsten Seite fortgesetzt.
\end{description}

%%%%%%%%%%%%%%%%%%%%%%%%%%%%%
\section{Hinweise zu bekannten Inkompatibilitäten}
\label{sec:Inkompatibilitäten}

\begin{description}
  \item[quote\{\}] Bei Verwendung des
  \verb|\quote{}|-Befehls entstehen Fehler im Layout. Daher ist als Ersatz die \verb|quotation|-Umgebung zu benutzen.
\end{description}

%%%%%%%%%%%%%%%%%%%%%%%%%%%%%
\section{Zusätzliche Pakete}
\label{sec:ZusätzlichePakete}

Die \LaTeX-Klasse \verb-ifathesis- wurde durch sinnvolle \LaTeX-Pakete erweitert, damit der Funktionsumfang für eine wissenschaftliche Arbeit angemessen ist. Wichtige und sinnvolle Pakete wurden in der Datei \verb-packages.tex- eingebunden. Die einzelnen Pakete sind dort mit einem kurzen Kommentar versehen. Eigene Paket-Erweiterungen können hier hinzugefügt werden.

Generell gilt: Benötigen Sie Befehle/Funktionen, die die Vorlage nicht zur Verfügung stellt, nutzen Sie ausgewählte Pakete und \emph{nicht} den nächstbesten Workaround der im Internet zu finden ist. Oft sind zusätzliche Pakete besser auf die verwendete Klasse und die anderen ebenfalls eingebundenen Pakete abgestimmt, als wenn man mit \TeX-Primitiven tief in die Grundlagen des Dokuments eingreift. Bei direkten Eingriffen können (nicht gleich erkennbare) Layoutfehler die Folge sein, da die Klasse und andere Pakete nicht "`wissen"' können, das Sie als Anwender etwas verändert haben (z.\,B. ein Textelement verschoben).

Lesen Sie auch undbedingt die Dokumentationen der bereits eingebundenen und von Ihnen hinzugefügten Packete um
\begin{itemize}
  \item einerseits über deren Möglichkeiten (z.\,B. die Befehle \verb-\toprule- und \verb-\bottomrule- bei Tabellen) und
  \item anderseits über Inkompatibilitäten oder Notwendigkeiten bei der Reihenfolge der Pakete Bescheid zu wissen!
\end{itemize}


%%%%%%%%%%%%%%%%%%%%%%%%%%%%%%%%%%%%%%%%%%%%%%%%%%%%%%%%%%%%%%%%%%%%%%
\section{Wichtige Informationen zur Form der Arbeit}
\label{sec:WichtigeInformationenZurFormDerArbeit}
%%%%%%%%%%%%%%%%%%%%%%%%%%%%%%%%%%%%%%%%%%%%%%%%%%%%%%%%%%%%%%%%%%%%%%

Text- und Formelsatz unterliegt typografischen Regeln, die im Allgemeinen nicht besonders bekannt sind. Dennoch ist es wichtig, sich an die Konventionen zu halten, damit das erstellte Dokument einem professionellen Anspruch genügt. Die wichtigsten Regeln und Hinweise findet man in wenigen guten Dokumenten zusammengefasst, die (neben weiteren nützlichen .pdf-Dateien) im Unterverzeichnis \verb-docs- der Vorlage zu finden sind. Bitte beachten Sie die die dort aufgestellten Regeln bei der Erstellung Ihrer Arbeit!

%%%%%%%%%%%%%%%%%%%%%%%%%%%%%%%%%%%%%%%%%%%%%%%%%%%%%%%%%%%%%%%%%%%%%%
%%%%%%%%%%%%%%%%%%%%%%%%%%%%%%%%%%%%%%%%%%%%%%%%%%%%%%%%%%%%%%%%%%%%%%
\chapter{IfA-Richtlinie f�r wissenschaftliche und studentische Arbeiten}
\label{sec:IfARichtlinieF�rWissenschaftlicheUndStudentischeArbeiten}


%%%%%%%%%%%%%%%%%%%%%%%%%%%%%%%%%%%%%%%%%%%%%%%%%%%%%%%%%%%%%%%%%%%%%%    
\section{Schriftliche Ausarbeitung}
\label{sec:SchriftlicheAusarbeitung}
%%%%%%%%%%%%%%%%%%%%%%%%%%%%%%%%%%%%%%%%%%%%%%%%%%%%%%%%%%%%%%%%%%%%%%

%%%%%%%%%%%%%%%%%%%%%%%%
\subsection{Allgemeines}
\label{sec:Allgemeines2}

Die Arbeit ist ohne Verzicht auf Vollst�ndigkeit kurz zu fassen. Der schriftliche Bericht soll dem Stand von Wissenschaft und Technik auf dem speziellen Gebiet entsprechen und die Durchf�hrung der Aufgabe, die mit ihr verkn�pften theoretischen und experimentellen Untersuchungen, ihre Ergebnisse und Schlussfolgerungen eindeutig und klar erkennen lassen. Jeder Teilabschnitt soll dem Lesenden Ergebnisse vermitteln, die er an einer anderen Stelle nicht oder zumindest nicht mit vergleichbar geringem Zeitaufwand findet. Es kommt darauf an, das Wesentliche der eigenen Arbeitsergebnisse entsprechend ihrem wissenschaftlichen Charakter klar, pr�gnant und einfach darzustellen. Zu empfehlen ist, am Schluss von gr��eren Abschnitten oder Teilproblemen die Ergebnisse gesondert zusammenzufassen.

Eine Richtlinie wichtiger Inhalte von Studien- und Diplomarbeiten finden Sie \href{http://www.et.tu-dresden.de/ifa/index.php?id=336&L=1?keepThis=true}{HIER}.

%%%%%%%%%%%%%%%%%%%%%%%%%%%%%%%%%%%%%%%%%%%%%%
\subsection{Zum allgemeinen Aufbau der Arbeit}
\label{sec:ZumAllgemeinenAufbauDerArbeit}

Folgende Reihenfolge wird empfohlen:
\begin{compactitem}
	\item Titelblatt
	\item Aufgabenstellung im Original (vom verantw. Hochschullehrer unterschrieben)
	\item Kurzfassung (deutsch)
	\item Abstract (englisch)
  	\item Inhaltsverzeichnis
  	\item Einleitung
  	\item Hauptteil (s.a. \href{http://www.et.tu-dresden.de/ifa/index.php?id=336&L=1?keepThis=true}{Inhaltsrichtlinie})
	\begin{compactitem}
		\item Anforderungsdefinition
	    \item L�sungskonzeption
    		\item Implementierungsbeschreibung
  		\item Verifikationsergebnisse
	\end{compactitem}
  	\item Zusammenfassung
  	\item Anh�nge
	\begin{compactitem}
			\item Anlagenverzeichnis
    			\item Anh�nge 1\dots n
	\end{compactitem}
  \item Literaturverzeichnis
  \item Selbstst�ndigkeitserkl�rung.
\end{compactitem}

Nachfolgende Anmerkungen erl�utern die Anstriche.


%-------------------------
\subsubsection{Titelblatt}
\label{sec:Titelblatt}

Als Titelblatt sind folgende Vorlagen zu verwenden:

\begin{compactitem}
	\item Titelblatt "'Diplomarbeit"' (\href{http://www.et.tu-dresden.de/ifa/fileadmin/user_upload/www_files/richtlinien_sa_da/DA-Deckblatt_IfA.doc}{doc-file}\footnote{Die LaTeX-Vorlage generiert bereits eine Titelseite entsprechend der Vorgaben.})
  	\item Titelblatt "'Studienarbeit"' (\href{http://www.et.tu-dresden.de/ifa/fileadmin/user_upload/www_files/richtlinien_sa_da/SA-Deckblatt_IfA.doc}{doc-file}\footnote{Die LaTeX-Vorlage generiert bereits eine Titelseite entsprechend der Vorgaben.})
\end{compactitem}


%-----------------------
\subsubsection{Textteil}
\label{sec:Textteil}

Der Textteil ist in dezimalklassifizierter Weise zu nummerieren. Die Gliederungstiefe sollte 4 Gliederungsebenen nicht �bersteigen. Das Inhaltsverzeichnis erh�lt keine Gliederungsnummer; die Einleitung ist der erste Gliederungsabschnitt (1 Einleitung). Die Einleitung beginnt auf einer neuen Seite.
Das Inhaltsverzeichnis beginnt auf einer neuen Seite. Das Wort "'Inhaltsverzeichnis"' ist als �berschrift zu schreiben. Das Verzeichnis muss die Abschnittsnummern, die Abschnitts�berschriften wie im Text sowie die jeweilige Seitenzahl (Beginn des Abschnittes) enthalten.
Das Inhaltsverzeichnis nennt nicht die Titelseite, die Aufgabenstellung, das Inhaltsverzeichnis selbst und die Selbst�ndigkeitserkl�rung.

Die Einleitung sollte folgende wesentliche Aussagen vermitteln:
\begin{compactitem}
	\item Einordnung in das Wissensgebiet
	\item Motivation f�r die Arbeit, Darstellung von Zusammenh�ngen, die zur Formulierung der Aufgabe gef�hrt haben
	\item Pr�zisierung der Aufgabenstellung, Vorgehensweise zur Probleml�sung.
\end{compactitem}

Im ausf�hrenden Textteil (aufgabenabh�ngig unterschiedlich viele Abschnitte) sind theoretische Grundlagen, L�sungsans�tze mit Bewertung m�glicher L�sungswege, ausgef�hrte L�sungen, Funktionsnachweise mit vollst�ndiger Beschreibung der Test- und Untersuchungsbedingungen darzustellen.
Der Umgang mit Bildern, Tabellen, Gleichungen und Literaturstellen ist im Abschnitt \ref{sec:FormDerArbeit} genauer erl�utert.
In der Zusammenfassung sind die wesentlichen Inhalte und Ergebnisse der Arbeit in �bersichtlicher Form zusammenzufassen und einer Wertung zu unterziehen. Gegebenenfalls sollte auf offene Probleme hingewiesen werden.
Am Beginn der Arbeit ist der Stand der Wissenschaft und Technik aufgabenbezogen darzustellen. Dar�ber hinaus muss in der gesamten schriftlichen Arbeit eine klare Abgrenzung des eigenen Beitrags von genutzten Vor- und Parallelarbeiten erfolgen. Dazu ist insbesondere die verwendete Literatur in einem Literaturverzeichnis zu zitieren.


%----------------------
\subsubsection{Anh�nge}
\label{sec:Anh�nge}

Als Anhang der wissenschaftlichen Arbeit sind solche Teile zu deklarieren, die wegen ihres gro�en Umfanges den Textteil der Arbeit sprengen w�rden, zum Verst�ndnis aber notwendig sind. Typische Inhalt von Anh�ngen sind Versuchsprotokolle, Simulationsprotokolle, detaillierte Darstellung technischer Realisierungen und Programmausdrucke.
Bei mehreren Anh�ngen ist ein Verzeichnis der Anh�nge voranzustellen.


%------------------------------------------
\subsubsection{Selbstst�ndigkeitserkl�rung}
\label{sec:Selbstst�ndigkeitserkl�rung}

Zur Best�tigung der selbstst�ndigen Anfertigung der Arbeit ist das entsprechende Muster zu verwenden:
\begin{compactitem}
	\item Selbstst�ndigkeitserkl�rung zur Diplomarbeit (\href{http://www.et.tu-dresden.de/ifa/fileadmin/user_upload/www_files/richtlinien_sa_da/DA_Selbststaendigkeit.doc}{doc-file})
  \item Selbstst�ndigkeitserkl�rung zur Studienarbeit (\href{http://www.et.tu-dresden.de/ifa/fileadmin/user_upload/www_files/richtlinien_sa_da/SA_Selbststaendigkeit.doc}{doc-file})
\end{compactitem}


%%%%%%%%%%%%%%%%%%%%%%%%%%%%
\subsection{Form der Arbeit}
\label{sec:FormDerArbeit}

%--------------------------
\subsubsection{Allgemeines}
\label{sec:Allgemeines3}

Studienarbeiten und Diplomarbeiten sind im Format DIN A4 vorzulegen. Gr��ere Bl�tter, z.B. Zeichnungen, sind auf dieses Format zu falten. Materialien in kleinerem Format sind auf Bl�tter vom Format DIN A4 aufzukleben oder in Taschen einzustecken. Nicht faltbare Materialien gr��er als DIN A4 sind als Band getrennt beizuf�gen. 

Tabelle \ref{tab:Seitennz�hlungUndReihenfolge} fasst die anzuwendenden Nummerierungsarten f�r die einzelnen Abschnitte der wissenschaftlichen Arbeit zusammen.

\begin{table*}[h]
	\centering
		\caption{Seitenz�hlung und Reihenfolge}
		\label{tab:Seitennz�hlungUndReihenfolge}
		\begin{tabular}{ll}
			\toprule
			Textteil									& Nummerierung \\
			\midrule
			Titelblatt	 							& keine \\
 			Aufgabenstellung	 						& keine \\
 			Kurzfassung (deutsch)					& keine \\
 			Abstract (englisch)	 					& keine \\
 			Inhaltsverzeichnis	 					& keine \\
 			Abbildungsverzeichnis					& keine \\
 			Tabellenverzeichnis 					& keine \\
 			Abk�rzungsverzeichnis 				& keine \\
 			Hauptteil (ab Einleitung)    			&	arabische Ziffern (beginned mit 1) \\
 			Anh�nge									& gesonderte Nummerierung empfohlen \\
 			Literaturverzeichnis	 					& keine \\
 			Selbstst�ndigkeitserkl�rung	 			& keine \\
 			\bottomrule
		\end{tabular}
\end{table*}

Die Abgabe der Studien- und Diplomarbeit hat in gebundener Form (Heftung, Spirale o.�.) sowie zus�tzlich als rechnerlesbare Datei (CD, DVD o.�.) zu erfolgen.


%----------------------------------
\subsubsection{Bilder und Tabellen}
\label{sec:BilderUndTabellen}

Als Bilder sind Fotos, Zeichnungen, Diagramme, Schemata u.a. zu benennen. Tabellen sind �bersichten, Aufz�hlungen, Gegen�berstellungen in numerischer oder textueller Form. Eine abschnittsweise Nummerierung ist empfehlenswert.
Bilder sind mit Bildunterschriften zu versehen (Bezeichnung "'Bild"', lfd. Nr., aussagef�hige Bezeichnung; z.B. Bild 3.1: Strecke mit PID-Regler).
Tabellen erhalten eine Tabellen�berschrift (Bezeichnung "'Tabelle"', lfd. Nr., aussagef�hige Bezeichnung; z.B. Tabelle 3.1: Aufwand bei Programmentwicklung).


%--------------------------
\subsubsection{Gleichungen}
\label{sec:Gleichungen}

Gleichungen sind entsprechend DIN 1338 (Formelschreibweise und Formelsatz) zu gestalten. Sie sind im Text fortlaufend zu nummerieren und mit runden Klammern am rechten Rand zu kennzeichnen. Eine abschnittsweise Nummerierung ist zu empfehlen. Gleichungen sind in den durchg�ngig lesbaren Text der Arbeit unter Beachtung von Interpunktionszeichen einzubeziehen. Formelzeichen sind zu erkl�ren. Der Verweis auf eine oder mehrere Gleichungen ist im Text mit der Abk�rzung Gl. (1.1) bzw. Gln. (1.1) zu beschreiben.

Bei der Wahl der Formelzeichen ist zu beachten, dass der Bezeichner $Ta$ in mathematischer Schreibweise $T \cdot a$ bedeutet. Deshalb ist in Formeln eine tiefgestellte Indizierung ($T_a$) zu bevorzugen. Sollen im Zusammenhang mit Rechenprogrammen gleichungsm��ige Zusammenh�nge erl�utert werden, so sind geeignete Indizes zu w�hlen (z.\,B. $T_a$) oder besondere Anmerkungen zu treffen.

\minisec{Beispiel}

...
\begin{align}
	a^2 + b^2 = c^2 \label{eq:pythagoras}
\end{align}
...
\begin{align}
	a_1 &= 15 \label{eq:constaeins} \\
	a   &= 4  \label{eq:consta} \\
	b   &= 2  \label{eq:constb}
\end{align}
... so erh�lt man unter Einbeziehung von Gl. \eqref{eq:pythagoras} und der Gln. \eqref{eq:constaeins} bis \eqref{eq:constb} schlie�lich die gew�nschte L�sung....


%-------------------------------
\subsubsection{Literaturstellen}
\label{sec:Literaturstellen}

Literaturstellen sind im Textteil durch gleiche K�rzel wie im Literaturverzeichnis in eckigen Klammern oder Schr�gstrichen / / anzugeben; bei w�rtlich zitierter Literatur sind au�erdem die Seiten anzugeben. Hilfe zum \verb-natbib--Paket ist \href{http://merkel.zoneo.net/Latex/natbib.php}{hier} zu finden.
Beispiele:
\begin{itemize}
	\item ... "`... und wie zu verfahren ist"' \citep[S. 211]{w3c_svg} ...
	\item ... \citet{gme_overview} beschreiben in ihrer Abhandlung...
	\item ... und ist bereits von verschiedenen Autoren beschrieben worden. \citep{xvcml-indin2007,frankel,steinberg_emf_2008} ...
\end{itemize}
M�gliche Ordnungsprinzipien im Literaturverzeichnis sind
\begin{itemize}
	\item numerische Ordnung in alphabetischer Reihenfolge nach jeweils erstem Autor
	\item numerische Ordnung nach der Reihenfolge der zitierten Quellen
	\item alphabetische Ordnung
\end{itemize}
Literaturangaben sind in der Vollst�ndigkeit und Interpunktion gem�� nachfolgender Beispiele (Buch, Zeitschrift, Tagungsband, Firmenschrift, Diplomarbeit, Standard, Internetquelle) auszuf�hren:
\renewcommand{\labelenumi}{[\theenumi]}
\begin{enumerate}
	\item Isermann, R.: Identifikation dynamischer Systeme, Band II. Springer-Verlag, Berlin, Heidelberg, New York, 1988.
	\item B�rmann, F.;  Greye, G.R.; L�deke, M.: Prozessregelung einer Nachreaktion auf der Basis eines k�nstlichen neuronalen Netzmodells. Automatisierungstechnische Praxis 37 (1995) 8, 36-43.
	\item Pereira, C.;  Rathke, T.: Objektorientierte Entwicklung von Echtzeitsystemen in der Automatisierungstechnik. Proc. 39. Int. Wiss. Kolloq., Ilmenau 1994.
	\item	National Instruments Corp.: LabView-Getting Started with LabView for Windows. Software-Dokumentation, 1992.
	\item Nitsche, R.: Entwurf und Erprobung eines Fuzzy-Reglers zur Reibkraftkompensation. Diplomarbeit FH Heilbronn, Feb. 1993.
	\item DIN 19227, Teil 2: Graphische Symbole und Kennbuchstaben f�r die Prozessleittechnik, Darstellung und Einzelheiten.
\end{enumerate}
\renewcommand{\labelenumi}{\theenumi.}
Alternativ kann auch die sog. Harvard-Nummerierung oder der Zitatstil des vorliegenden Dokuments (siehe Literaturverzeichnis) verwendet werden.

Bei der Harvard-Nummerierung wird ein K�rzel aus drei Buchstaben (des einzigen bzw. der ersten drei Autoren) und zwei Ziffern des Erscheinungsjahres gebildet:
\begin{list}{[AAA]}{
		\setlength{\leftmargin}{2.6cm} 
		\setlength{\labelwidth}{2cm} 
		\renewcommand{\makelabel}[1]{[#1]}
	}
	\item[Bra-95]	Branicky, M.S.: Studies in hybrid systems: Modeling, analysis and control. Diss. Massachusetts Institute of Technology, Cambridge 1995.
	\item[B�h-88]	B�hler, H.: Anti-Reset-Windup-Ma�nahmen bei stetigen Reglern. at - Automatisierungstechnik 36 (1988) 5, 190-191.
	\item[CEO-93]	Cellier, F.E.; Elmqvist, H.; Otter, M. u.a.: Guidelines for Modeling and Simulation of Hybrid Systems. IFAC World Congress. Sydney 1993, 1219-1225.
	\item[Eng-97]	Engell, S.: Modellierung und Analyse hybrider dynamischer Systeme. at - Automatisierungstechnik 45 (1997) 4, 152-161.
	\item[Fil-60]	Filippov, A.F.: Differential equation with discontinuous right-hand sides. Mathematicheskii Sbornik 51 (1960).
	\item[SSc-00]	Schaft, A.v.d.; Schumacher, H.: An introduction to hybrid dynamic systems. (Lecture notes in control and information Science 251) London, Berlin, Heidelberg (Springer) 2000.
	\item[Sur-04]	SurTec Deutschland GmbH: SurTec 680 Chromitierungsberechnungen; URL: \href{http://berechnung.surtec.com/Chromitierung/}{http://berechnung.surtec.com/Chromitierung/} (Stand: 27.07.2004).
	\item[TWM-00]	Thiele, W.; Wildner, K.; Matschiner, H. u.a.: Offenlegungsschrift OS DE 198 50 530 A. Kreislaufverfahren zum Beizen von Kupfer und Kupferlegierungen (2000).
	\item[ZUt-96]	Zhao, F.; Utkin V.: Adaptive Simulation and Control of Variable-structure Control Systems in Sliding Regimes. Automatica 32 (1996) 7, 1037-1042.
\end{list}
Achtung: Fehlendes Kenntlichmachen von Zitaten kann zur Nichtannahme einer wissenschaftlichen Arbeit f�hren!


%%%%%%%%%%%%%%%%%%%%%%%%%%%%%%%%%%%%%%%%%%%%%%%%%%%%%%%%%%%%%%%%%%%%%%    
\section{Anzahl der Exemplare, Abgabe der Arbeit}
\label{sec:AnzahlDerExemplareAbgabeDerArbeit}
%%%%%%%%%%%%%%%%%%%%%%%%%%%%%%%%%%%%%%%%%%%%%%%%%%%%%%%%%%%%%%%%%%%%%%

F�r die Herstellung der Originale und Kopien ist der Bearbeiter verantwortlich!

Wenn durch den Hochschullehrer nicht anders festgelegt, ist nachfolgend genannte Anzahl von Exemplaren abzugeben.

\minisec{Studienarbeiten}
\begin{compactitem}
	\item 2 Exemplare (gedruckt)
	\item 1 CD (mit elektronisch lesbarer Form der Studienarbeit, der Kurzfassung, des Abstracts, den verwendeten Bildern
				sowie zur Arbeit geh�renden Programm- und Daten-Files)
	\item Kurzfassung und Abstract (jeweils ein gedrucktes Exemplar)
	\item Abgabemodus f�r Studienarbeiten
		\begin{compactitem}
			\item Der Student legt dem Betreuer/verantw. Hochschullehrer termingerecht zwei Exemplare der Studienarbeit zur Best�tigung der Vollst�ndigkeit vor (Signum und Datum auf Deckblatt).
			\item Der Student gibt anschlie�end beide Exemplare sowie Kurzfassung bzw. Abstract im {\bf Sekretariat (Frau M�ge, BAR E04 bzw. Frau Kindermann, BAR E23)} ab und erh�lt den Laufzettel (\href{http://www.et.tu-dresden.de/ifa/fileadmin/user_upload/www_files/richtlinien_sa_da/DA-SA_Laufzettel.pdf}{pdf-file}) des Instituts f�r Automatisierungstechnik.
			\item Das Sekretariat �bergibt die Exemplare an den Betreuer zur Weiterleitung an die Gutachter.
	\end{compactitem}
\end{compactitem}

\minisec{Diplomarbeiten}
\begin{compactitem}
	\item 1 CD (mit elektronisch lesbarer Form der Diplomarbeit, der Kurzfassung, des Abstracts, den verwendeten Bildern sowie zur Arbeit geh�renden Programm- und Daten-Files)
	\item Kurzfassung und Abstract (jeweils ein gedrucktes Exemplar)
	\item Poster (zur Verteidigung)
	\item 2 Exemplare (gedruckt)
	\item Abgabemodus f�r Diplomarbeiten:
		\begin{compactitem}
			\item Zuerst legt der Diplomand dem Betreuer/verantw. Hochschullehrer termingerecht ein Exemplar der Diplomarbeit zur Best�tigung der Vollst�ndigkeit vor (Signum und Datum auf Deckblatt).
			\item Danach bringt der Diplomand das vom Hochschullehrer signierte Exemplar in das Pr�fungsamt der Fakult�t Elektrotechnik und Informationstechnik zur Registrierung und Best�tigung.
			\item Im Anschluss werden das signierte Exemplar und das zweite Exemplar der Arbeit sowie Kurzfassung/Abstract im Sekretariat {\bf (Frau M�ge, BAR E04 bzw. Frau Kindermann, BAR E23)}  abgegeben und dort der Laufzettel (\href{http://www.et.tu-dresden.de/ifa/fileadmin/user_upload/www_files/richtlinien_sa_da/DA-SA_Laufzettel.pdf}{pdf-file}) des Instituts f�r Automatisierungstechnik entgegengenommen.
		\end{compactitem}
\end{compactitem}

Das Sekretariat �bergibt die Exemplare an den Betreuer zur Weiterleitung an die Gutachter.


%%%%%%%%%%%%%%%%%%%%%%%%%%%%%%%%%%%%%%%%%%%%%%%%%%%%%%%%%%%%%%%%%%%%%%  
\section{Kurzfassung/Abstract}
\label{sec:KurzfassungAbstract}
%%%%%%%%%%%%%%%%%%%%%%%%%%%%%%%%%%%%%%%%%%%%%%%%%%%%%%%%%%%%%%%%%%%%%%  

Die Kurzfassung (engl.: Abstract) gibt auf einer Seite DIN A4 einen zusammenfassenden �berblick �ber die Arbeit. Die Formulierung ist so zu halten, dass auch f�r Nichtspezialisten die Inhalte und Anwendungsaspekte der Arbeit zug�nglich sind (Problembeschreibung/-einordnung, neue L�sungsans�tze, Systemeigenschaften durch die neuen Ans�tze, Anwendungsfelder). Die verbalen Aussagen sind durch ein charakteristisches Bild zur Thematik zu erg�nzen.
Die Kurzfassungen (deutsch/englisch) sind rechnerlesbar (Diskette/CD z.B. in MS Word) sowie als pdf-Ausdruck (je 1-fach) im IfA-Sekretariat einzureichen.

\minisec{Mustervorlage}
\begin{compactitem}
	\item Kurzfassung/Abstract (\href{http://www.et.tu-dresden.de/ifa/fileadmin/user_upload/www_files/richtlinien_sa_da/DA_SA_Kurzfassung_Abstract.doc}{doc-file})
\end{compactitem}

\minisec{Beispiel}
\begin{compactitem}
	\item \href{http://www.et.tu-dresden.de/typo3/ifa/index.php?id=710}{Studien- und Diplomarbeiten }
\end{compactitem}


%%%%%%%%%%%%%%%%%%%%%%%%%%%%%%%%%%%%%%%%%%%%%%%%%%%%%%%%%%%%%%%%%%%%%%  
\section{Poster (nur f�r Diplomarbeit)}
\label{sec:PosterNurFuerDiplomarbeit}
%%%%%%%%%%%%%%%%%%%%%%%%%%%%%%%%%%%%%%%%%%%%%%%%%%%%%%%%%%%%%%%%%%%%%%  

Zur Ver�ffentlichung der wichtigsten Arbeitsergebnisse der Diplomarbeit im Posterschaukasten des Instituts f�r Automatisierungstechnik ist ein Poster zu gestalten und zur Verteidigung der Diplomarbeit vorzulegen. Gestaltungsrichtlinien sind im Anhang \ref{sec:Postergestaltung} dieser Empfehlungen enthalten. Die Ma�angaben sind unbedingt verbindlich.


%%%%%%%%%%%%%%%%%%%%%%%%%%%%%%%%%%%%%%%%%%%%%%%%%%%%%%%%%%%%%%%%%%%%%%  
\section{Verteidigung}
\label{sec:Verteidigung}
%%%%%%%%%%%%%%%%%%%%%%%%%%%%%%%%%%%%%%%%%%%%%%%%%%%%%%%%%%%%%%%%%%%%%%  

Diplomarbeiten und Studienarbeiten werden �ffentlich verteidigt. In diesen Verteidigungen h�lt der Kandidat einen Vortrag �ber Ziele, Inhalt und Ergebnisse seiner vorgelegten Arbeit. Der Vortrag sollte maximal 30 Minuten dauern. Projektionsm�glichkeiten f�r Folien sollten genutzt werden (Projektor/Beamer stehen zur Verf�gung). Anschlie�end erfolgt eine Diskussion, zu der der Kandidat ein Schlusswort halten kann.
Im �brigen gelten die Bestimmungen der entsprechenden Diplompr�fungsordnung (siehe Pr�fungsamt ET).

\minisec{Besonderheiten bei Diplomarbeiten}
\begin{compactitem}
	\item Der Student legt zur Verteidigung ein Poster zur Diplomarbeit sowie den vollst�ndig unterschriebenen IfA-Laufzettel vor.
	\item Nach der Verteidigung erh�lt der Student durch Unterschrift des Hochschullehrers auf dem Formblatt "'Exmatrikulation"' die Best�tigung �ber die erfolgte Verteidigung.
	\item Die Diplomakte wird erst nach Vorlage des ordnungsgem�� ausgef�llten IfA-Laufzettels an das Pr�fungsamt weitergeleitet.
\end{compactitem}

\minisec{Besonderheiten bei Studienarbeiten}
\begin{compactitem}
	\item Der Student legt zur Verteidigung den vollst�ndig unterschriebenen IfA-Laufzettel vor.
	\item Das Pr�fungsergebnis wird erst nach Vorlage des ordnungsgem�� ausgef�llten IfA-Laufzettels an das Pr�fungsamt weitergeleitet.
\end{compactitem}

\end{document}
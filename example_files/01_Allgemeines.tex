%%%%%%%%%%%%%%%%%%%%%%%%%%%%%%%%%%%%%%%%%%%%%%%%%%%%%%%%%%%%%%%%%%%%%%
%%%%%%%%%%%%%%%%%%%%%%%%%%%%%%%%%%%%%%%%%%%%%%%%%%%%%%%%%%%%%%%%%%%%%%
\chapter{Allgemeine Hinweise zur Vorlage}
	\label{sec:AllgemeineHinweiseZurVorlage}


\section{Installation}
Die Vorlage \verb-ifathesis- ist eine fertige \LaTeX-Klasse, die die Richtlinie f�r wissenschaftliche Arbeiten am \emph{Institut f�r Automatisierungstechnik, Technische Universit�t Dresden} realisiert. Um diese Vorlage zu verwenden m�ssen die folgenden Dateien lediglich im Wurzelverzeichnis des zu verfassenden \LaTeX-Dokuments liegen:
\begin{compactitem}
	\item \verb-ifathesis.cls-
	\item \verb-packages.tex-
	\item \verb-kopfzeile.{pdf, eps}-
\end{compactitem}
Alternativ k�nnen diese Dateien dem \LaTeX-Paketbaum hinzugef�gt werden. Dann steht die Vorlage stets zur Verf�gung. Die daf�r notwendigen Schritte sind allerdings von der verwendeten \LaTeX-Distribution sowie vom Betriebssystem abh�ngig, daher wird an dieser Stelle auf eine Internetrecherche verwiesen: \href{http://www.google.de/search?sourceid=chrome&ie=UTF-8&q=latex+install+packages}{z.B. �ber Google.}

\section{Konfiguration}
Dieser Vorlage liegt ein umfassendes Beispiel bei, welches mit der Vorlage erstellt wurde (siehe \verb-example.tex-). 

\subsection{Dokumentenklasse, Dokumentenoptionen}

Zur Verf�gung steht die Dokumentenklasse \verb-ifathesis- mit folgenden Optionen:

\begin{description}
	\item[print | screen] �bersetzt das Dokument optimiert f�r den Druck bzw. f�r das Lesen am Monitor (einseitig, farbig hervorgehobene Links). \textbf{Achtung: Die Gr��e des Textk�rpers unterscheidet sich geringf�gig. Daher bitte immer zuerst das Dokument f�r den Druck optimert erstellen, wodurch bspw. die Gr��e der Grafiken richtig bestimmt werden kann. Anschlie�end kann in die f�r das Monitorlesen optimierte Version �bersetzt werden}\footnote{Zur Abgabe der Arbeit auf einem Datentr�ger erforderlich}. Voreinstellung: \verb-print-. 
	\item[listoffigures] Integriert ein Abbildungsverzeichnis an der richtigen Stelle, gem�� der \emph{ifa}-Richtlinie
	\item[listoftables] Integriert ein Tabellenverzeichnis an der richtigen Stelle, gem�� der \emph{ifa}-Richtlinie
\end{description}

\subsection{Weitere Parameter}

\begin{description}
	\item[ifaThesis] Art der Arbeit (g�ltige Werte \verb-Diplomarbeit, Studienarbeit-)
	\item[ifaAuthor] Name des Autors
	\item[ifaAuthorBirthday] Geburtsdatum des Autors (Format: \verb-TT.MM.JJJJ-)
	\item[ifaAuthorBirthplace] Geburtsort des Autors
	\item[ifaAuthorCourse] Studiengang des Autors
	\item[ifaAuthorYearOfMatriculation] Immatrikulationsjahr des Autors
	\item[ifaKeywords] Schl�sselw�rter, die dem Thema zugeordnet werden k�nnen (als kommaseparierte Liste)
	\item[ifaTitle] Titel der Arbeit
	\item[ifaSupervisor{\emph X}] mit $X=A\dots E$, Angabe von bis zu f�nf Betreuern (A\,--\,E).
	\item[ifaProfessor] Betreuender Hochschullehrer
	\item[ifaDayOfSubmission] Tag der Einreichung
	\item[ifaAppendix] Dateiname der Hauptdatei der Anh�nge (vollst�ndige Pfadangabe ausgehend von der Dokumentenwurzel erforderlich)
	\item[ifaReferences] Name der Bibliographiedatei (vollst�ndige Pfadangabe ausgehend von der Dokumentenwurzel erforderlich)
	\item[ifaSelfRelianceDeclaration] Dateiname der Selbst�ndigkeitserkl�rung (vollst�ndige Pfadangabe ausgehend von der Dokumentenwurzel erforderlich)
\end{description}

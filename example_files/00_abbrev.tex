%*******************************************************************************
% * Copyright (c) 2006-2010 
% * Institute of Automation, Dresden University of Technology
% * 
% * All rights reserved. This program and the accompanying materials
% * are made available under the terms of the Eclipse Public License v1.0 
% * which accompanies this distribution, and is available at
% * http://www.eclipse.org/legal/epl-v10.html
% * 
% * Contributors:
% *   Institute of Automation - TU Dresden, Germany 
% *      - initial API and implementation
% ******************************************************************************/

% Dieser Befehl setzt das Nomenklaturverzeichnis mit angegebener Breite f�r die erste Spalte
\printnomenclature[2.3cm]

% Falls das Abk�rzungs- und Symbolverzeichnis ohne Nutzung des nomencl-Pakets 
% von Ihnen selber erstellt werden soll (z.B. in einer Tabelle), ersetzen Sie 
% einfach den /printnomenclature - Befehl durch eigenen Quellcode.

% In Ihrem Dokument k�nnen Sie dann bei jeder Verwendung einer Abk�rzung oder 
% eines neuen Symbols diese/dieses kurz �ber den \nomenclature Befehl erkl�ren 
% und dem Abk�rzungsverzeichnis zur Verf�gung stellen.

% Ein Beispiel f�r ein Symbol ("yx"):
   \nomenclature[yx ]{$a_{BA}$}{Anteil der ausgegebenen Nullen}

% Ein Beispiel f�r eine Abk�rzung ("ba"):
%   \nomenclature[ba ]{DIP}{Bildprozessor ("`Digital Image Processor"')}


%%%%%%%%%%%%%%%%%%%%%%%%%%%%%%%%%%%%%%%%%%%%%%%%%%%%%%%%%%%%%%%%%%%%%%
%%%%%%%%%%%%%%%%%%%%%%%%%%%%%%%%%%%%%%%%%%%%%%%%%%%%%%%%%%%%%%%%%%%%%%
\chapter{Verbindlichkeiten vorab}
\label{sec:VerbindlichkeitenVorab}

\begin{compactitem}
  \item Verbindliche Grundlage ist die zu Beginn der Diplom-/Studienarbeit g�ltige IfA-Richtlinie.
  \item Die \href{http://www.et.tu-dresden.de/ifa/fileadmin/user_upload/www_files/richtlinien_sa_da/wiarb_97.pdf}{Empfehlung der Fakult�t Elektrotechnik und Informationstechnik f�r die Ausarbeitung wissenschaftlicher Arbeiten} (Studienarbeiten oder Diplomarbeiten) ist als Erg�nzung gedacht, im Konfliktfall ist die IfA-Richtlinie anzuwenden.
  \item Diplom-/Studienarbeit sind in Absprache mit dem Betreuer gem�� dem \href{http://www.et.tu-dresden.de/ifa/index.php?id=331&L=1?keepThis=true}{IfA-Vorgehensmodell} abzuwickeln.
  \item F�r die Inhalte einer Studien-/Diplomarbeit gilt die \href{http://www.et.tu-dresden.de/ifa/index.php?id=336&L=1?keepThis=true}{Richtlinie} des Instituts f�r Automatisierungstechnik, die als Beispieltext auch in Kapitel \ref{sec:IfARichtlinieF�rWissenschaftlicheUndStudentischeArbeiten} dieses Dokuments zu finden ist
\end{compactitem}
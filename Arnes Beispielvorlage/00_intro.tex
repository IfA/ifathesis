\begin{titlepage}
	\begin{picture}(0,0)
	\put(-91,-50){\includegraphics*{./bilder/kopfzeile}}
	\end{picture}
	
	\begin{center}
		\vfill
		
		{\huge\textsc{Diplomarbeit}}
		\vfill
		
		zum Thema \\[0.5cm]
			
		{\large \LaTeX-Vorlage f�r Diplomarbeiten am Institut f�r Automatisierungstechnik der Technischen Universit�t Dresden}
		\vfill
		
		\begin{tabular}{rl}
			vorgelegt von 	& Max\,Mustermann\\
			geboren am 			& 02.06.1983 in Dresden\\ 
		\end{tabular}
		\vfill
		
		{\large zur Erlangung des akademischen Grades} \\[0.2cm]
		
		{\LARGE Diplomingenieur\\
						(Dipl.-Ing.)}
		\vfill
		
		\begin{tabular}{rl}
			Betreuer: 												& Dipl.-Ing.\,\,B.\,Treuer,\\
																				& Dr.-Ing.\,\,S.\,Onstiger\\
			Verantwortlicher									& \\
			Hochschullehrer: 									& Prof.\,Dr.\,techn.\ K.\,Janschek\\
			Tag der Einreichung: 							& 7. August 2008\\
		\end{tabular}
	\end{center}
\end{titlepage}

\clearpage
\begingroup
	\pagestyle{empty}	%Im Vorspann nur Seiten ohne Kopf- und Fu�zeile (au�er Abschnittsbeginn)
	\renewcommand*{\chapterpagestyle}{empty} %Auch Abschnittsbeginn ohne Kopf- und Fu�zeile

	%\includepdf[offset=15pt 9pt]{bilder/aufgabenstellung.pdf}
	%\includepdf[offset=15pt 0]{bilder/kurzfassung.pdf}
	%\includepdf[offset=15pt 0]{bilder/abstract.pdf}
	\tableofcontents
	\clearpage
	\listoffigures
	\clearpage
	\listoftables
	\clearpage
	
%	%Symbole und Abk�rzungen normalerweise im Text definiert, da wo sie gebraucht werden. Hier au�nahmsweise vorher:
%	\nomenclature[yx ]{$a_{BA}$}{Anteil der ausgegebenen Nullen (kein Speicherzugriff)}
%	\nomenclature[yx ]{$a_{BE}$}{Anteil der eingef�gten Nullen (kein Speicherzugriff)}
%	\nomenclature[yx ]{$a_{SA}$}{Anteil der geschriebenen Werte (Speicherzugriff)}
%	\nomenclature[yx ]{$a_{SE}$}{Anteil der eingelesenen Werte (Speicherzugriff)}
%	\nomenclature[yx ]{$f_{FPGA}$}{Taktfrequenz des FPGA}
%	\nomenclature[yx ]{$\tilde{f}_{\text{Praes}}$}{gesch�tzte Bearbeitungsfrequenz im Pr�sentationsmodus (d.\,h. ohne Vorhersage)}
%	\nomenclature[yx ]{$\tilde{f}_{\text{Praez}}$}{gesch�tzte Bearbeitungsfrequenz im Pr�zisionsmodus (d.\,h. mit Vorhersage)}
%	\nomenclature[yx ]{$f(k,l)$}{Originalsignal (Signalbereich)}
%	\nomenclature[yx ]{$F(m,n)$}{Originalsignal (Frequenzbereich)}
%	\nomenclature[yx ]{$g(k,l)$}{Originalsignal (Signalbereich)}
%	\nomenclature[yx ]{$G(m,n)$}{Originalsignal (Frequenzbereich)}
%	\nomenclature[yx ]{$G^*(m,n)$}{Originalsignal, konjugiert komplex (Frequenzbereich)}
%	\nomenclature[yx ]{$k$}{Laufvariable Korrelationsfunktion, Zeilenrichtung}
%	\nomenclature[yx ]{$l$}{Laufvariable Korrelationsfunktion, Spaltenrichtung}
%	\nomenclature[yx ]{$m$}{Laufvariable diskretes Signal, Zeilenrichtung}
%	\nomenclature[yx ]{$M$}{Gr��e eines diskreten Segments, Zeilenrichtung}
%	\nomenclature[yx ]{$n$}{Laufvariable diskretes Signal, Spaltenrichtung}
%	\nomenclature[yx ]{$N$}{Gr��e eines diskreten Segments, Spaltenrichtung}
%
%	\nomenclature[ba ]{1D}{eindimensional(e)}
%	\nomenclature[ba ]{2D}{zweidimensional(e)}
%	\nomenclature[ba ]{A}{Anforderung}
%	\nomenclature[ba ]{Altera\TReg}{FPGA Hersteller}
%	\nomenclature[ba ]{Base-TX}{Ethernetspezifikation}
%	\nomenclature[ba ]{DIP}{Bildprozessor ("`Digital Image Processor"')}
%	\nomenclature[ba ]{E/A}{Ein(gang) / Aus(gang)}
%	\nomenclature[ba ]{EPP}{Parallelportspezifikation ("`Enhanced Parallel Port"')}
%	\nomenclature[ba ]{FFT}{Fast-Fourier-Transformation}
%
%	%\markboth{\nomname}{\nomname}
%	
%	%Dieser Befehl setzt an diese Stelle das Nomenklaturverzeichnis.
%	\printnomenclature[2.3cm]
\endgroup
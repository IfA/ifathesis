\chapter{Einleitung}
\label{sec:einleit}

\color{Red}
Achtung: Diese Sammlung von verschiedenen Dateien soll als Vorlage f�r Diplomarbeiten am Institut f�r Automatisierugnstechnik dienen. Rot gedruckte Textstellen sind Anmerkungen des Autors und sollten aufmerksam gelesen werden. Der weitere Text dient als L�ckenf�ller und als Beispiel. Durch K�rzungen und Umstellungen ist der Sinn des schwarzen Beispieltextes ver�ndert und nicht mehr garantiert! Bitte die enthaltenen Beispielabbildungen und Beispieltexte nicht ohne Zustimmung des Autors Arne Sonnenburg kopieren, weitergeben oder ver�ffentlichen.

Auch die Positionierung der Bilder wurde nach der K�rzung noch nicht optimiert, wie das in einer richtigen Arbeit mit Hilfe des Befehls \verb|\clearpage| oder \verb|cleardoublepage| gemacht werden sollte.
\color{Black}

"`Dem Einsatz von Robotern erschlie�t sich in den letzten Jahren ein immer gr��eres Anwendungsfeld. Sei es die klassische Fabrikautomation, der Servicebereich (mobile Roboter), oder die Unterhaltungsindustrie. Voraussetzung f�r solche Anwendungen ist die F�higkeit des Roboters sich im Raum orientieren zu k�nnen."' \cite{technische_universitat_wien_fortgeschrittene_2007}.

Eine dynamische, unbekannte Umgebung, wie sie im Allgemeinen anzutreffen ist, stellt dabei besondere Anforderungen an den Roboter: Die Navigation und die Hindernisvermeidung ist aufw�ndig.

"`Mit Hilfe klassischer Sensoren, wie etwa Infrarot-, Ultraschall- oder taktiler Sensoren, ist dies nur sehr bedingt m�glich, vor allem dann, wenn sich der Roboter in einer komplexen, unstrukturierten Umgebung orientieren soll."' \cite{technische_universitat_wien_fortgeschrittene_2007}.

Leistungsf�higere Sensoren, wie zum Beispiel Laserscanner oder Stereokamera basierte Abstandssensoren, sind meist zu teuer, um im Niedrigpreissektor eingesetzt zu werden. Daher wird der Visuellen Navigation zunehmend Aufmerksamkeit geschenkt, besonders seit kosteng�nstige Webcams als Massenprodukt verf�gbar sind \cite{nguyen_xuan_dao_visual_2005}.

Diese Arbeit entwirft und untersucht einen FPGA-basierten Prozessor, der diesen Optischen Fluss aus einer Bildsequenz bestimmt. Dabei baut sie auf einer am Institut f�r Automatisierungstechnik der Technischen Universit�t Dresden angefertigten Diplomarbeit auf, die unter anderem die Bestimmung des Optischen Flusses mit einem PC untersucht hat \cite{zaunick_echtzeitgenerierung_2006}.

Die Motivation f�r diese Arbeit ist, dass ein bereits bekanntes und untersuchtes, f�r praktische Anwendungen robustes Verfahren vom PC auf ein preisg�nstiges, handels�bliches FPGA-Board ausgelagert werden kann. Dies hat die Vorteile, dass
\begin{compactitem}[\hspace{1.5em}--]
	\item die Rechenleistung zum Bestimmen des Optischen Flusses auf dem PC f�r andere Berechnungen zur Verf�gung steht und somit die Echtzeitf�higkeit zunimmt,
	\item sich das Board �ber eine Standardschnittstelle sowohl an Desktop-PCs als auch an Notebooks anschlie�en l�sst,
	\item ein Framegrabber nicht mehr n�tig ist und
	\item die Bearbeitungsgeschwindigkeit im Vergleich zum PC erh�ht wird.
\end{compactitem}

Die Arbeit ist folgenderma�en aufgebaut: Im Kapitel \ref{sec:stand} wird kurz dargestellt, wie sich die Berechnung des Optischen Flusses in der Forschung und Anwendung einordnen l�sst. In Kapitel \ref{sec:anfdef} sind die Nuzteranforderungen erarbeitet und das geforderte System wird funktional analysiert. Die Variantendiskussion ist in Kapitel \ref{sec:vardisk} zu finden. Hier werden M�glichkeiten diskutiert und festgelegt, den Optischen Fluss Prozessor zu realisieren. Darauf aufbauend, enth�lt Kapitel  \ref{sec:entwurf} die Beschreibung des Entwurfs und der Implementierung. Der Entwurf wird in Kapitel \ref{sec:valid} auf die Erf�llung der Nutzeranforderungen und die korrekte Implementierung �berpr�ft. Die Arbeit schlie�t in Kapitel \ref{sec:ausblick} mit einer Zusammenfassung der Ergebnisse, deren Vor- und Nachteilen, sowie Vorschl�gen zur weiteren Verbesserung.
